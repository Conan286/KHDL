\makeglossaries
%\makenoidxglossaries
\newlength{\glscolwidthA}
\newlength{\glscolwidthB}
\setlength{\glscolwidthA}{2cm} % Độ rộng cột name
\setlength{\glscolwidthB}{\dimexpr\linewidth-\glscolwidthA-4\tabcolsep} % Độ rộng cột description

\newglossarystyle{mylist}{%
    \setglossarystyle{long}% sử dụng longtable style
    \renewenvironment{theglossary}%
        {\begin{longtable}{p{\glscolwidthA}p{\glscolwidthB}}}%
        {\end{longtable}}%
    \renewcommand*{\glossaryheader}{}%
    \renewcommand*{\glsgroupheading}[1]{}%
    \renewcommand*{\glsgroupskip}{}%
    \renewcommand{\glossentry}[2]{%
        \glstarget{##1}{\glossentryname{##1}} & \glossentrydesc{##1}\tabularnewline
    }%
}


% Danh mục thuật ngữ và từ viết tắt
\newglossaryentry{iaas}{
    type=\acronymtype,
    name={IaaS: Infrastructure  as  a  Service},
    description={Dịch vụ hạ tầng},
    first={Dịch vụ hạ tầng}
}
\newglossaryentry{API}{
    type=\acronymtype,
    name={API: Application Programming Interface},
    description={Giao diện lập trình ứng dụng},
    first={API}
}
\newglossaryentry{EUD}{
    type=\acronymtype,
    name={EUD: End-User Development},
    description={Phát triển ứng dụng người dùng cuối},
    first={End-User Development}
}
\newglossaryentry{GWT}{
    type=\acronymtype,
    name={GWT: Google Web Toolkit},
    description={Công cụ lập trình Javascript bằng Java của Google},
    first={Công cụ lập trình Javascript bằng Java của Google (Google Web Toolkit}
}
\newglossaryentry{HTML}{
    type=\acronymtype,
    name={HTML: HyperText Markup Language},
    description={Ngôn ngữ đánh dấu siêu văn bản},
    first={Dịch vụ hạ tầng (Infrastructure  as  a  Service - IaaS)}
}