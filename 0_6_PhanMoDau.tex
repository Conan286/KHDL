\begin{center}
    \textbf{\Large PHẦN MỞ ĐẦU}
\end{center}
\textbf{Mục tiêu và định hướng cá nhân về quá trình thực tập cơ sở}: Trong quá trình thực tập cơ sở, mục tiêu của tôi là củng cố những kiến thức đã học và phát triển khả năng ứng dụng công nghệ vào các bài toán thực tế. Một trong những định hướng của tôi là phát triển các hệ thống thông minh, hỗ trợ người dùng tìm kiếm thông tin nhanh chóng và chính xác. Dự án này đóng vai trò quan trọng trong việc cung cấp giải pháp cho ngành du lịch, cụ thể là đề xuất khách sạn tại Hà Nội dựa trên nhu cầu của người dùng. Lý do tôi chọn đề tài này là vì du lịch đang trở thành ngành phát triển mạnh tại Việt Nam, và công nghệ thông tin có thể giúp tối ưu hóa trải nghiệm của du khách.

\textbf{\textit{Trình bày phần đặt vấn đề liên quan đến đề tài của TTCS: }} Công nghiệp 4.0 đang thúc đẩy sự phát triển mạnh mẽ của các ứng dụng công nghệ trong mọi lĩnh vực của cuộc sống. Ngành du lịch và dịch vụ cũng không nằm ngoài xu hướng này, đặc biệt trong việc tối ưu hóa trải nghiệm khách hàng. Trong bối cảnh đó, hệ thống gợi ý khách sạn thông minh dựa trên nhu cầu của người dùng có thể giúp du khách dễ dàng tìm kiếm và lựa chọn nơi lưu trú phù hợp.

\textbf{\textit{Trình bày phần các giải pháp hiện tại và hạn chế: }}  Hiện tại, các hệ thống gợi ý khách sạn phổ biến như Booking.com hay Agoda đã cung cấp nhiều lựa chọn phong phú, nhưng phần lớn dựa trên đánh giá người dùng hoặc xếp hạng sao. Những phương pháp này có thể không hoàn toàn phản ánh được nhu cầu thực sự của từng cá nhân trong từng thời điểm. Một số du khách vẫn phải tìm kiếm thủ công, gây mất thời gian.

\textbf{\textit{Trình bày phần mục tiêu và hướng giải pháp:}}  Để giải quyết vấn đề này, hệ thống đề xuất khách sạn thông minh dựa trên input từ người dùng sẽ giúp tự động gợi ý các khách sạn phù hợp nhất. Hệ thống này không chỉ giúp du khách tiết kiệm thời gian tìm kiếm mà còn mang lại trải nghiệm tốt hơn, cá nhân hóa hơn.

\textbf{\textit{Trình bày phần đóng góp của Đồ án và bố cục của Đồ án:}}  Hướng tới nhu cầu đó, đề tài đồ án tốt nghiệp có tên “\textbf{\textit{Ứng dụng nhận dạng ký tự quang cho bài toán nhận diện biển số xe}}”.

Nội dung trình bày trong báo cáo gồm 3 chương chính: 
\begin{itemize}
    \item Chương 1: Báo cáo tiến độ từng tuần trong quá trình training 
    \item Chương 2: Đề xuất và báo cáo đề tài Thực tập cơ sở bao gồm 3 nội dung:
    \begin{itemize}
        \item Giới thiệu chung: Trình bày tổng quan về đề tài, xác định mục tiêu xây dựng hệ thống đề xuất khách sạn tại Hà Nội dựa trên thông tin người dùng nhập vào. Đối tượng của hệ thống là du khách cần tìm kiếm nơi lưu trú tại Hà Nội. Phương hướng giải quyết là xây dựng một hệ thống gợi ý khách sạn thông minh, cá nhân hóa dựa trên thông tin như vị trí, ngân sách, loại hình khách sạn mong muốn. Hệ thống sẽ sử dụng các thuật toán gợi ý và phân tích dữ liệu để đưa ra kết quả tối ưu nhất. Giới thiệu các kiến thức liên quan về hệ thống gợi ý, xử lý dữ liệu và giao diện người dùng.
        \item Cơ sở lý thuyết: Tìm hiểu cơ sở lý thuyết về hệ thống gợi ý, bao gồm các thuật toán như collaborative filtering, content-based filtering, và hybrid recommendation systems. Phân tích các đặc điểm của ngành khách sạn tại Hà Nội như phân khúc giá, địa điểm, dịch vụ. Lựa chọn các thư viện và công cụ như Python, Pandas, Scikit-learn để tiến hành xây dựng mô hình gợi ý và ReactJS cho giao diện người dùng. 
    \item Xây dựng chương trình: Trình bày sơ đồ hệ thống, bao gồm các bước từ thu thập dữ liệu, tiền xử lý dữ liệu, xây dựng mô hình gợi ý, và phát triển giao diện người dùng. Sau đó, tiến hành đóng gói hệ thống và tích hợp vào website demo. Triển khai thực nghiệm, kiểm tra tính hiệu quả của hệ thống và đánh giá kết quả so với các phương pháp gợi ý truyền thống.
    \end{itemize}
    \item Chương 3: Kết luận quá trình Thực tập bao gồm Bài học và kết quả đạt được từ đó rút ra những điều cần cải thiện trong tương lai.
\end{itemize}

